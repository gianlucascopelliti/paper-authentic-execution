
\newcounter{lemma}
\setcounter{lemma}{0}

% -----------------------------------------------------------------------------
% -----------------------------------------------------------------------------
\subsection{Formal Definition of our Security Properties}
\label{sec:secprops}
% -----------------------------------------------------------------------------

% -----------------------------------------------------------------------------
\subsubsection{Programs.}
\label{language}
% -----------------------------------------------------------------------------
%
For the purpose of this paper, we consider event-driven programs. To
precisely define our security property, we define a simple formal reactive
programming language, the syntax of which shown in
Figure~\ref{fig:syntaxmodellang}. We emphasize that our implementation
supports the full C language for writing event handlers.

\begin{figure}
  \begin{tabular}{p{0.5\textwidth} p{0.5\textwidth}}
    \begin{minipage}{0.5\textwidth}
      \centering
      \begin{tabular}{rrl}
        $m$ & $\Coloneqq$ & $\textrm{module\ }id( id*;id*);  h*$ \\
        $h$ & $\Coloneqq$ & $\textrm{on } id(x) \: \{c\}$ \\
        $c$ & $\Coloneqq$ & $\textrm{skip}$ \\
        & $\mid$ & $c; c$ \\
        & $\mid$ & $g\coloneqq e$ \\
        % & $\mid$ & $r\coloneqq \textrm{declassify } e$ \\
        & $\mid$ & $\textrm{if } e \textrm{ then } \{ c_1 \} \textrm{ else } \{ c_2 \}$ \\
        & $\mid$ & $\textrm{while } e \: \{c\}$ \\
        & $\mid$ & $id(e)$ \\
      \end{tabular}
    \end{minipage} &
    \begin{minipage}{0.5\textwidth}
      \centering
      \begin{tabular}{rrl}
        $e$ & $\Coloneqq$ & $x \mid n \mid g \mid e \odot e$ \\
        $\odot$ & $\Coloneqq$ & $+ \mid - \mid \; = \; \mid \; <$ \\
        $cn$ & $\Coloneqq$ & $id \to id $ \\
        $p$ & $\Coloneqq$ & $m* ; cn* ;$ \\
        $o$ & $\Coloneqq$ & $ !id(n) \mid \cdot $ \\
        $i$ & $\Coloneqq$ & $  ?id(n) $ \\
        $\alpha$ & $\Coloneqq$ & $ o \mid i$
      \end{tabular}
    \end{minipage} \\
  \end{tabular}
  \caption{Syntax of our event-driven language.}
  \label{fig:syntaxmodellang}
\end{figure}

A \emph{\reactmod} $m$ declares a name,  zero or more identifiers for input
channels, and zero or more identifiers for output channels.  For each
declared input channel, it then defines exactly one event handler ($h$)
which  handles events arriving on that input channel.  Events carry data
(in our simple language only integers), and on arrival of the event, the
received integer is bound to the formal parameter $x$. The handler then
executes its body, a command $c$. Commands are standard and can use
assignment to module global variables $g$ to share information between the
handlers in the same module. Module global variables do not need to be
declared and are initialized to~0.

A program ($p$) consists of zero or more modules, and zero or more
connections ($cn$), where a connection $id_1 \to id_2$ specifies that the
outputs sent on output channel $id_1$ should be delivered to input channel
$id_2$. For simplicity, we assume that all input and output channels have
unique names in the entire program (in practice, one would want to have
module-scoped namespaces). Each output channel is connected to at most one
input channel, and each input channel is connected to at most one output
channel. The input and output channels that are not connected to anything
are the {\em primitive} input and output channels of the application. We
write $p(\overline{in};\overline{out})$ to say that $p$ is a program with
primitive input channels $\overline{in}$ and primitive output channels
$\overline{out}$.

\Cref{fig_model} shows (parts of) the code of the $\app{Vio}$ application
described in~\cite{noorman:authentic-execution}.  The
code implementing the event handler for the parking sensor on $\node{P1}$ stores in the global
variable \verb+taken+ whether there is currently a car on the parking spot.
If so, the event handler for a clock tick on $\node{P1}$ will increase a counter. If
that counter exceeds MAX, it sends a violation event. To define the input
and output channels of the \app{Vio} program, we write
$\app{Vio}(P1,T1,P2,T2;Display)$.

The semantics of programs is as expected: execution is triggered
by an event on a primitive input channel. We write such input events ($i$)
as $ ?id(n) $.  This leads to the execution of the corresponding handler
for that event. Execution of this handler generates  output events $o$.
These can be internal (unobservable, silent) events that we write as
$\cdot$, or primitive output events that we write as  $ !id(n) $. 

Below we specify the formal semantics for programs as a labelled transition system, where labels are events $\alpha$, thus defining exactly what traces $\overline{\alpha}$ of events a program allows.
The following is a trace of the $\app{Vio}$ program (with \texttt{MAX == 2}, showing only primitive events):
%
\[
   ?P1(1), ?T1(0), ?T1(0),
   ?T1(0), !Display(1)
\]
%
After the third  input event on $clock_{NP1}$, the \module{\textit{VioP1}}
module will send a $Violation(1)$ output that generates a $Violation1(1)$ input event on
\module{\textit{VioD}}, which then displays that parking spot~1 has been
taken for too long.  Note that the events on $Violation$ and $Violation1$ are internal events and do not show up in the trace.
%
Note that in general, execution of programs is non-deterministic -- when one handler generates output events connected to different input channels, the semantics does not impose an order in
which the resulting input events must be handled. 

We introduce the following notation for traces: given a trace $\overline{\alpha}$, we write $\alpha\downarrow_{\overline{id}}$ for the trace obtained from $\alpha$ but keeping only 
the events $?id_i(n)$ and $!id_i(n)$ with $id_i \in \overline{id}$.


% -----------------------------------------------------------------------------
\subsubsection{Security Properties of Modules.}
% -----------------------------------------------------------------------------
%
%\todo{Reviewer 1: I appreciate the attention to replay and reordering
%attacks.}
The compilation and deployment scheme for our event-driven distributed
applications is described in detail in~\cite{noorman:authentic-execution}.
We now discuss the security guarantees that this compilation and deployment of modules gives us.
Consider a source module $m(\overline{in},\overline{out})$, and the
corresponding \protmod{}  $\compiled{m}$.
The operational semantics of $m$ defines what traces of source level events of the form $?id(n)$ and $!id(n)$ this module has.
At run-time, $\compiled{m}$ will only perform the following two kinds of events:
\begin{itemize}
  \item  Output events, where the module hands a new output to the event manager.
         We use the notation $!(id,nc,n)$ for  an event where the module calls \handleoutput{} on the connection $id$ with nonce $nc$ and payload $n$.
  \item Input events, where the event handler hands an event to \handleinput{} and this event passes the authenticity check.
        We use the notation $?(id,nc,n)$ for an event where \handleinput{} successfully accepts an event on connection $id$ with nonce $nc$ and payload $n$.
\end{itemize}

We use the notation $\uncompiled{ }$ to relate run-time events to corresponding source-level events, i.e. $\uncompiled{!(id,nc,n) } = !id(n)$  and $\uncompiled{?(id,nc,n)} = ?id(n)$ .

The security guarantee that we obtain is the following. Assume that $m$ has been correctly compiled, deployed and initialized (i.e. SetKey has been called by the deployer for
every connection), resulting in $\compiled{m}$.
%
\begin{lemma} \label{lem:mod}
If $\compiled{m}$ has at run-time a trace $\rho$ of run-time events, then $m$ has trace $\uncompiled{\rho}$ of source-level events.
\end{lemma}
%
The lemma is proven in the Section~\ref{sec:proofs}, but the intuition is the following.
After initialization is completed, the isolation properties of the TEE ensure that the only
interactions that are possible with $\compiled{m}$ are in-calls to its entry points (\handleinput{} and \setkey), and out-calls from within the module itself. From the construction of $\compiled{m}$ as described above, it follows that both entry points drop anything that does not pass the authenticity check,
and that the only out-call the module ever performs is to \handlelocalevent{} with a correctly constructed output message.
So $\compiled{m}$ has no other interactions with its environment than the two types of events defined above.
That these events correspond to a trace of source-level events allowed by the operational semantics of the module then follows from our assumption that the $\compiled{m}$ is created from the $m$ using a correct compiler.

% -----------------------------------------------------------------------------
\subsubsection{Security Properties of Programs.}
\label{formal-security-properties}
% -----------------------------------------------------------------------------
%
Now, consider a program $p(\overline{in},\overline{out})$, and the deployed program $\compiled{p}$ at the end of phase 1 of deployment.
Any further execution of $\compiled{p}$ (independent of whether deployment proceeds with phase 2) satisfies the following.
%
\begin{lemma}\label{lem:prog}
If $\compiled{p}$ has a run-time trace $\rho$, then $p$ has a trace $\uncompiled{\rho}$.
\end{lemma}
%
The proof is again in Section~\ref{sec:proofs}, but the intuition is the following: according to \cref{lem:mod} each individual module behaves as its source counterpart. For the
connections between modules, we know that there is a unique key unknown to the attacker per connection, and the use of nonces (sequence numbers) and authenticated encryption
with this key is sufficient to turn the untrusted network in a queue: only the module that outputs on the connection can put the appropriate cryptographic messages on the 
network, and the module reading from the connection is assured of authenticity and correct order of receipt.

%\todo{Reviewer 1: it is not clear
%whether physical compromise of a single device is tolerable}
%We are now ready to state the main security theorem.
%
Let $p(\overline{in},\overline{out})$ be a program with primitive inputs $\overline{in}$ and primitive outputs $\overline{out}$. 
Deployment connects $in_i$ to physical
input channels $pi_i$ and
$out_j$ to physical output channels $po_j$. 
%
We use the notation $?pi(n)$ for the occurrence of a physical input event on $pi$, and $!po(n)$ for the
occurrence of a physical output event on $po$. These physical events are
the events we care about, and that can be observed or that will have effects in the
real world at run-time. We extend the notation $\uncompiled{ }$  in the obvious way (e.g. $\uncompiled{?pi(n)} = ?id(n)$ when deployment connects $id$ to $pi$).

Consider the time frame starting at the end of phase 2a of deployment, and ending at a point where the deployer starts a new attestation of the protected driver module
for $po_j$. Suppose that  this attestation eventually succeeds, and that the deployer has observed a sequence $\rho_{out_j}$ of outputs on $po_j$. Then the following holds.
% 
\begin{theorem}
$p$ has a trace $\overline{\alpha}$ such that (1) $\overline{\alpha}\downarrow_{out_j} = \uncompiled{\rho_{out_j}}$,
and (2) for each primitive input channel $in_i \in \overline{in}$, there has been a contiguous sequence of inputs $\rho_{in_i}$ on $pi_i$ with  $\overline{\alpha}\downarrow_{in_i} = \uncompiled{\rho_{in_i}}$.
\end{theorem}
% 
The proof is given in the Section~\ref{sec:proofs}, but the intuition is: since $\rho_{out_j}$ is observed on an output device that was attested to only accept authenticated output events from $p$,
there must have been an execution and hence a trace $\rho$ of $p$ with
these output events. Lemma~\ref{lem:prog} allows us lift this to a trace of the source program.
Since $\compiled{p}$ will only respond to authenticated sequences of inputs from the $pi_i$, there must have been such contiguous sequences of inputs.


% -----------------------------------------------------------------------------
% -----------------------------------------------------------------------------
\subsection{Formal Semantics for our Model Language}
\label{sec:semantics}
% -----------------------------------------------------------------------------

In Section~\ref{language}, we introduced a simple reactive programming,
the syntax of which is shown in Figure~\ref{fig:syntaxmodellang}.  This
section defines the operational semantics of this language using standard
techniques.

% -----------------------------------------------------------------------------
\subsubsection{Modules.}
% -----------------------------------------------------------------------------
%
We first define how modules are executed.
A module configuration $(\mu, c)$ consists of a  (module-private) store $\mu$,  and the currently executing command~$c$.
%
We use the following semantic judgements:

\begin{itemize}
\item $\mu \vdash e \Downarrow n$ says that the ground expression $e$ evaluates under the variable store $\mu$ to the integer $n$. A {\em ground} expression does not contain
any parameters $x$, but it can contain global variables $g$. A store $\mu$ is a mapping from global variable names $g$ to integers $n$.
The definition of this judgement is completely standard and hence omitted.
\item $(\mu,c) \argtrans{\alpha}(\mu',c')$ says that module configuration $(\mu,c)$ can step to $(\mu',c')$ by performing transition $\alpha$.
A transition $\alpha$ can be an output $!id(n)$, or it can be an input $?id(n)$, i.e. the occurrence of an  event on input channel $id$.
A transition $\alpha$ can also be an unobservable (silent, module-internal) action denoted by $\cdot$. For technical reasons, we use the metavariable $o$ for outputs and silent
transitions, and $i$ for inputs.


The definition of this judgment for a specified module $m$ is given in Figure~\ref{fig:semhandler}, where the function $\LOOKUP(m,id)$ looks up the body of the handler for $id$ in the module $m$.
We use the notation $\mu[g \mapsto n]$ for an update of the store $\mu$ that sets global variable $g$ to $n$, and the notation $c[x\leftarrow n]$ for
the substitution of  integer $n$ for formal parameter $x$ in command $c$.
\end{itemize}

\begin{figure}[h]
	\begin{equation}
	\frac{}{(\mu, \textrm{skip}; c) \silenttrans (\mu, c)}
	\end{equation}
	\begin{equation}
	\frac{(\mu, c_1) \argtrans{o} (\mu', c_1')}{(\mu, c_1; c_2) \argtrans{o} (\mu', c_1'; c_2)}
	\end{equation}
	\begin{equation}
	\frac{\mu \vdash e \Downarrow n}{(\mu, g \coloneqq e) \silenttrans (\mu[g \mapsto n], \textrm{skip})}
	\end{equation}
	\begin{equation}
	\frac{c = \textrm{if } e \textrm{ then } \{c_1\} \textrm{ else } \{c_2\} \quad \mu \vdash e \Downarrow n \quad n \neq 0}{(\mu, c) \silenttrans (\mu, c_1)}
	\end{equation}
	\begin{equation}
	\frac{c = \textrm{if } e \textrm{ then } \{c_1\} \textrm{ else } \{c_2\} \quad \mu \vdash e \Downarrow 0}{(\mu, c) \silenttrans (\mu, c_2)}
	\end{equation}
	\begin{equation}
	\frac{c = \textrm{while } e \: \{ c_{\textrm{loop}} \} \quad \mu \vdash e \Downarrow 0}{(\mu, c) \silenttrans (\mu, \textrm{skip})}
	\end{equation}
	\begin{equation}
	\frac{c = \textrm{while } e \: \{ c_{\textrm{loop}} \} \quad \mu \vdash e \Downarrow n \quad n \neq 0}{(\mu, c) \silenttrans (\mu, c_{\textrm{loop}}; c)}
	\end{equation}
	\begin{equation}
	\frac{\mu \vdash e \Downarrow n}{(\mu, id(e)) \argtrans{!id(n)} (\mu, \mathrm{skip})}
	\end{equation}
	\begin{equation}
	\frac
		{\LOOKUP(m,id) = c}
		{(\mu, \textrm{skip}) \argtrans{?id(n)} (\mu, c[x \leftarrow n])}
	\end{equation}
	\caption{Small-step semantics for modules.}
	\label{fig:semhandler}
\end{figure}


The {\em initial state} of a module is $(\mu_0, \textrm{skip})$ where $\mu_0$ maps every global variable to the integer $0$.
%
We say a module $m$ has a trace $\overline{\alpha}$ if, starting from its initial state, it can do all the transitions in $\overline{\alpha}$.

% -----------------------------------------------------------------------------
\subsubsection{Programs.}
% -----------------------------------------------------------------------------
%
We can now turn to the definition of the semantics of programs.
A program configuration $\overline{(\mu,c)} ; \overline{in : q}$ consists of a sequence of module configurations $\overline{(\mu,c)}$ (discussed above),
and a sequence of connection queues $\overline{in : q}$ that associate a queue of natural numbers with each input channel (this queue
contains the messages that have already been sent to this input channel but not yet received from that channel).

The initial state of a program is the state: $\overline{(\mu_0,\mathrm{skip})}, \overline{in : []}$  (i.e. an initial module state for each module in the program, and
all input channels start with the empty queue).
%
The rules in Figure~\ref{fig:semprogram} define the semantics of programs.

\begin{figure*}[htbp]
	\begin{equation} \tag{E-RECV}
	\frac
		{ (\mu_i, c_i) \argtrans{?in_j(n)}(\mu_i',c_i' ) }
		{\overline{(\mu,c)}, (\mu_i, c_i); \overline{in : q}, in_j: q_j :: n \silenttrans \overline{(\mu,c)},(\mu_i',c_i' ); \overline{ in : q}, in_j:q_j }
	\end{equation}
	\begin{equation} \tag{E-SILENT}
	\frac
		{ (\mu_i, c_i) \silenttrans (\mu_i',c_i' ) }
		{\overline{(\mu,c)}, (\mu_i, c_i); \overline{in : q} \silenttrans \overline{(\mu,c)},(\mu_i',c_i' ); \overline{ in : q} }
	\end{equation}
	\begin{equation} \tag{E-SEND}
	\frac
		{ (\mu_i, c_i) \argtrans{!id(n)}(\mu_i',c_i' ) \quad id \to in_j }
		{\overline{(\mu,c)}, (\mu_i, c_i); \overline{in : q}, in_j: q_j \silenttrans \overline{(\mu,c)},(\mu_i',c_i' ); \overline{ in : q}, in_j: n::q_j }
	\end{equation}
	\begin{equation} \tag{E-INPUT}
	\frac
		{  \not\to in_i }
		{\overline{(\mu,c)}; \overline{in : q}, in_i: q_i \argtrans{?in_i(n)} \overline{(\mu,c)}; \overline{ in : q}, in_i: n::q_i }
	\end{equation}
	\begin{equation} \tag{E-OUTPUT}
	\frac
		{ (\mu_i, c_i) \argtrans{!id(n)}(\mu_i',c_i' ) \qquad id \not\to}
		{\overline{(\mu,c)}, (\mu_i, c_i); \overline{in : q} \argtrans{!id(n)} \overline{(\mu,c)},(\mu_i',c_i' ) ; \overline{ in : q} }
	\end{equation}
	\caption{Small-step semantics for programs.}
	\label{fig:semprogram}
\end{figure*}

Rule (E-RECV) describes the case where one of the modules consumes an event from its (non-empty) input queue. Rule (E-SILENT) describes the case where one of the modules
does an internal computation step. Rule (E-SEND) describes the case where one of the modules performs an output, and the output channel is connected to input channel $in_j$.
Hence, the output value is put in the queue corresponding to $in_i$. Rule (E-INPUT) describes an input on a primitive input channel $in_i$ (i.e. $in_i$ is not connected to any
output channel of the program). Such an input is just buffered in the appropriate queue, and no modules change state. The input will at a later point in time be removed from the
queue by rule (E-RECV). Finally, rule (E-OUTPUT) describes the occurrence of a primitive output event.

We say a program $p$ has a trace $\overline{\alpha}$ if, starting from its initial state, it can do all the transitions in $\overline{\alpha}$.


% -----------------------------------------------------------------------------
% -----------------------------------------------------------------------------
\subsection{Proofs of Lemmas and Theorem 1.}
\label{sec:proofs}
% -----------------------------------------------------------------------------

We model \protmods{} at run-time as states in a labeled transition system. We call the set of states $S$, and we denote the transition relation as
$s_1 \argtrans{\beta} s_2$, with $\beta$ a run-time event. Given the
isolation properties of \protmods{}, the only run-time interactions that are possible
with a \protmod{} are (1) calls into an entry-point, and (2) out-calls from within the module to code outside.

For the \protmods{} that we construct in the paper, there are only two entry points: \setkey{} and \handleinput. \setkey{} is only called once by
the deployer during initialization, and has no effect on later calls. Hence, for the \protmods{},
it suffices to consider the following run-time events:
\begin{itemize}
\item Output events, where the \protmod{} hands a new output to the event manager. We use the notation $!(id,nc,n)$ for  an event where the module calls \handleoutput{} on
the connection $id$ with nonce $nc$ and payload $n$.
\item Input events, where the event handler hands an event to \handleinput{} and where this event passes the authenticity check. We use the notation $?(id,nc,n)$ for an event where
\handleinput{} successfully accepts an event on connection $id$ with nonce $nc$ and payload $n$.
\item Internal computation within the \protmod{}. We use the notation $\cdot$ for these silent steps.
\end{itemize}

Let $\compiled{m}$ be the initial state of the \protmod{} after compilation and deployment (including the appropriate calls to \setkey) of source module $m$.
We relate run-time events to corresponding source-level events using the notation $\uncompiled{!(id,nc,n) } = !id(n)$  and $\uncompiled{?(id,nc,n)} = ?id(n)$ ,
and we use the same notation on traces $\uncompiled{\overline{\beta}} = \overline{\alpha}$ which maps the $\uncompiled{}$ function over the elements of the trace.

\begin{lemma} \label{lem:mod2}
If $\compiled{m}$ has at run-time a trace $\overline{\beta}$ of run-time events, then $m$ has trace $\uncompiled{\overline{\beta}}$ of source-level events.
\end{lemma}

Proof:
We model the fact that $\compiled{m}$ is created by a functionally correct and trusted compiler, by assuming that there exists a function (also denoted $\uncompiled{}$), that
relates a run-time module state $s$ to a source module configuration $\uncompiled{s} = (\mu,c)$, and by assuming that this function is a backward simulation.
I.e. we assume that (1)  $\uncompiled{\compiled{m}} = (\mu_0,\mathrm{skip})$ (initial state of the compiled module maps to an initial source module configuration), and (2) if
$s \argtrans{\overline{\beta}} s'$, then$\uncompiled{s} \argtrans{\overline{\alpha}} \uncompiled{s'}$ with $\uncompiled{\overline{\beta}} \downarrow_{\overline{in},\overline{out}}
= \overline{\alpha}  \downarrow_{\overline{in},\overline{out}}$ ( if a run-time state $s$ can make a number of steps to state $s'$, then the corresponding source module configuration can
also make a number of steps with corresponding observable transitions, and ending in a source module configuration corresponding to $s'$).
From this formalization of the assumption of a correct module compiler, and
from the observation that the state of \protmods{} can not be affected in any other way than
by the run-time events $\beta$ (thanks to the \ac{SM}), the lemma follows directly. \hfill $\square$

Next let $p(\overline{in},\overline{out})$ be a source program. We assume all input and output channels used in the program have different identifiers, and use $mod(id)$ to
denote the module that has channel $id$. We also assume
that the deployer has chosen connection identifiers $cn$ for each connection in the program, and we define $in(cn)$ to be the input channel identifier and $out(cn)$ to be
the output channel identifier associated to connection $cn$. 

For channels $\overline{in}$ and $\overline{out}$ that are not connected in the program (and hence that will be connected to physical inputs and outputs in the deployment descriptor), we assume that the
deployer configures them with a key $K_{in_i}$ (respectively $K_{out_j}$)
that authenticates messages from (respectively to) a driver
\protmod{}.

We model the untrusted network at run-time as two sets of messages $(M,M_{all})$, where $M$ are the message currently still available on the network, and $M_{all}$ is the set of
all messages that have ever been put on the network. The attacker can remove messages from $M$, and can add messages to $M$ (and hence to $M_{all}$) under the
constraints of the Dolev-Yao model (i.e. he can not forge cryptographic messages for which he does not have the key).
For the execution of $p$, we only care about the valid cryptographic messages on the network, i.e. all messages of the form $K_{cn}(nc,n)$  created with connection key $K_{cn}$  (or keys $K_{in_i}$ or $K_{out_j}$)
and with nonce $nc$ and payload $n$. Hence, we restrict the $M$ and $M_{all}$ to just these messages (any other messages the attacker might put on the network can not
influence the execution of $p$).

Now, the run-time state of a program $p$ consists of:

\begin{itemize}
  \item The run-time states for all of its modules; and
  \item The state of the network.
\end{itemize}

The initial state is $\compiled{p} =
(\overline{\compiled{m}},(\{\},\{\}))$, and run-time events are the
run-time events $\beta$ of each of the \protmods{}, interleaved with attacker
steps where the attacker manipulates the network as discussed above.

Given a run-time state $s$ of one of the modules $m$ in $p$, we use the notation $nonce(cn,s)$ for the value of the nonce associated with connection id $cn$ in $s$.

\begin{lemma}\label{lem:prog2}
If $\compiled{p}$ has a run-time trace $\overline{\beta}$, then $p$ has a trace $\uncompiled{\overline{\beta}}$.
\end{lemma}
Proof: We construct a backward simulation from run-time program states to program configurations in the source level semantics.
We define $\uncompiled{(\overline{s},(M,M_{all}))}$ to be the program configuration $(\overline{\uncompiled{s}}; \overline{id:q})$ where (1) run-time module
states $s$ are mapped on $\uncompiled{s}$ using the backward simulation on module states that we get from the correct-compiler-assumption, and (2) 
the $q_i$ are defined as follows: (keep in mind that these are queues in the source level program configuration that contain the values that have been sent to some input channel $in$ but
not yet received from that input channel): 
\begin{itemize}
\item if $in$ is connected with connection identifier $cn$ (i.e. there exists a $cn$ such that $in(cn) = in$), then
let $nc$ be $nonce(cn,s)$ where $s$ is the run-time state of $mod(in)$. Hence $nc$ tells us how many events have already been received from that input channel. Now, construct the largest sequence of messages in $M_{all}$ that
has the following form:
$$
K_{cn}(nc,n_0), K_{cn}(nc+1,n_1), K_{cn}(nc+2,n_2), \ldots
$$
Then queue $q_i = n_0::n_1::n_2::\ldots$.
\item if $in$ is not connected in $p$ then it will later be connected to physical input channel, and the deployer configures it with a key $K_{in}$. 
$q_i$ is constructed in a similar way, but using $K_{in}$ instead of $K_{cn}$.
\end{itemize}
For the remainder of the proof, we have to show that $\uncompiled{}$ is a backward simulation. This is proven  by considering all possible run-time steps. We go over some
of the interesting cases:
\begin{itemize}
\item If the attacker does a step, this only impacts $M$, and hence the source program configuration can stay constant to match this. (The backward simulation does not depend on $M$)
\item Any silent step by any of the modules can be matched with zero or
more (E-SILENT) steps at source level. This follows directly from the fact
that $\uncompiled{}$ on \protmod{} states is a backward simulation (\cref{lem:mod2}).
\item If one of the run-time modules performs a $!(id,nc,n)$ event, where $id $ is connected in the program, then a new valid cryptographic message appears on the network (in $M$ and in $M_{all}$). It is easy to check that the source program can match this with an (E-SEND) step.
\item If a driver \protmod{} connected to $in$ during deployment sends a new physical input,  then a new valid cryptographic message appears on the network (in $M$ and in $M_{all}$). It is easy to check that the source program can match this with an (E-INPUT) step.
\item If one of the run-time modules performs a $?(id,nc,n)$ event, where $id $ is connected in the program, then the nonce in the runtime state of the receiving module will go up with 1.
This will remove an element from the corresponding queue in the backward simulation.
It is easy to check that the source program can match this with an (E-RECV) step. 
\item If one of the run-time modules performs a $!(id,nc,n)$ event, where
$id $ is not connected in the program, then a new valid cryptographic
message appears on the network (in $M$ and in $M_{all}$) constructed with
the key $K_{out}$ that authenticates this message to a driver \protmod{} for a physical output device.  It is easy to check that the source program can match this with an (E-OUTPUT) step.
\end{itemize}
This completes the proof. \hfill $\square$

Finally, we turn to the security theorem.
Let $p(\overline{in},\overline{out})$ be a program with primitive inputs $\overline{in}$ and primitive outputs $\overline{out}$. 
Deployment connects $in_i$ to physical
input channels $pi_i$ and
$out_j$ to physical output channels $po_j$. 

We use the notation $?pi(n)$ for the occurrence of a physical input event on $pi$, and $!po(n)$ for the
occurrence of a physical output event on $po$. We extend the notation $\uncompiled{ }$  in the obvious way (e.g. $\uncompiled{?pi(n)} = ?id(n)$ when deployment connects $id$ to $pi$).

Consider the time frame starting at the end of phase 2a of deployment, and
ending at a point where the deployer starts a new attestation of the driver
\protmod{} 
for $po_j$. Suppose that  this attestation eventually succeeds, and that the deployer has observed a sequence $\rho_{out_j}$ of outputs on $po_j$. Then the following holds.
% 
\begin{theorem}
$p$ has a trace $\overline{\alpha}$ such that (1) $\overline{\alpha}\downarrow_{out_j} = \uncompiled{\rho_{out_j}}$,
and (2) for each primitive input channel $in_i \in \overline{in}$, there has been a contiguous sequence of inputs $\rho_{in_i}$ on $pi_i$ with  $\overline{\alpha}\downarrow_{in_i} = \uncompiled{\rho_{in_i}}$.
\end{theorem}
Proof:
The driver \protmod{} for $po_j$ was correctly deployed at the start of the time-frame, and is verified to still be correctly deployed at the end of the time-frame.
It follows that it was deployed correctly for the entire time-frame, as an
attacker can only destroy the state of a \protmod{} (for instance by rebooting the system),
but can not redeploy the \protmod{} again afterwards.

Since $po_j$ was in a good state the entire time-frame, the physical outputs observed on $po_j$ must have been caused by by run-time events $!(out_j,nc,n)$ of the program.
Now consider the entire trace $\overline{\beta}$ of run-time events of the program up to and including the last of these $!(out_j,nc,n)$ events.

For this trace $\overline{\beta}$, we know from \cref{lem:prog2} that $p$ has $\overline{\alpha} = \uncompiled{\overline{\beta}}$. 
From the construction of $\overline{\alpha}$, it follows that $\overline{\alpha}\downarrow_{out_j} = \uncompiled{\rho_{out_j}}$.

Now, for each $in_i$, the corresponding \protmod{} $\compiled{mod(in_i)}$ will have performed zero or more $?(in_i,nc,n)$ events. From (1) the construction of that module,
and (2) from the fact that the deployer connects it to a driver \protmod{}
using a key $K_{in_i}$ that is shared only between $mod(in_i)$ and the
driver \protmod{} 
of $pi_i$, and (3) the assumption that driver \protmods{} reliably turn $po_i$ events into authenticated messages to $in_i$, we can conclude that there must have been
a contiguous sequence of  of inputs $\rho_{in_i}$ on $pi_i$  with  $\overline{\alpha}\downarrow_{in_i} = \uncompiled{\rho_{in_i}}$.

\hfill $\square$

