\sclist{0.5}{%
  \lstinputlisting[language=Python]{setkey.py}}{}%
  {Pseudocode of the \setkey{} entry point. Note that \unwrap{} uses the module
  key to decrypt the payload and throws an exception if the operation failed
  (i.e., the payload's \acs{MAC} is incorrect).\label{code:setkey}}%
  {code:setkey}
%
%
\sclist{0.5}{%
  \lstinputlisting[language=Python]{handleinput.py}}{}%
  {Pseudocode of the \handleinput{} entry point. Erroneous accesses to the
  tables as well as errors during \unwrap{} cause exceptions. Thus, these
  events, as well as those for which no input key has been set, are ignored.
  \unwrap{} takes a key and the expected associated data as arguments.}%
  {code:handleinput}
%
%
\sclist{0.5}{%
  \lstinputlisting[language=Python]{handleoutput.py}}{}%
  {Pseudocode of the generated output wrapper. Since the compiler generates
  calls to this function that cannot be called from outside the module, the
  connection identifier is always valid and no error checking is
  necessary.}%
  {code:output} 
%
%
\sclist{0.5}{%
  \lstinputlisting[language=Python]{attest.py}}{}%
  {Pseudocode of the \attest{} entry point. Since the attestation of an
  \protmod{} is \ac{TEE}-specific, we use a high-level function \genattevid{} to
  retrieve the attestation evidence. The actual implementation of this function
  depends on the \ac{TEE} used.}%
  {code:attest} 