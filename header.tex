\usepackage{mathtools}
\usepackage{listings}
\usepackage{listings-rust}
\usepackage{listings-yaml}
\usepackage{endnotes}
\usepackage[utf8]{inputenc}
\usepackage[english]{babel}
\usepackage{microtype}
\usepackage{amsmath}
\usepackage{bold-extra}
\usepackage{xstring}
\usepackage{color, colortbl}
\usepackage{booktabs}
\usepackage{adjustbox}
\usepackage[english=american,babel=false]{csquotes}
\usepackage{array}
\usepackage{caption}
\usepackage{subcaption}
\usepackage{hyperref}
\usepackage{xspace}
\usepackage{paralist}
\usepackage{longtable}

\usepackage[capitalize]{cleveref}
\crefname{figure}{Fig.}{Figs.}
\Crefname{figure}{Fig.}{Figs.}
\crefname{table}{Tab.}{Tabs.}
\Crefname{table}{Tab.}{Tabs.}
\crefname{section}{Sect.}{Sects.}
\Crefname{section}{Sect.}{Sects.}

\usepackage{tikz}
\usepackage{pgfplots}
\pgfplotsset{
    width=5cm, 
    legend style={
        font=\footnotesize,
        %rounded corners=2pt
    },
    compat=newest
}
\usepackage[utf8]{inputenc}
\DeclareUnicodeCharacter{2212}{−}
\usepgfplotslibrary{groupplots,dateplot}
\usetikzlibrary{calligraphy, calc, arrows.meta, decorations.pathreplacing, matrix, positioning, fit, backgrounds, patterns,shapes.arrows, shapes.symbols}

\usepackage{acronym}

% colour scheme for colour blind people
\definecolor{color1}{RGB}{254,97,0}
\newcommand{\coloriname}{orange}
\definecolor{color2}{RGB}{255,176,0}
\newcommand{\coloriiname}{yellow}
\definecolor{color3}{RGB}{100,143,255}
\newcommand{\coloriiiname}{blue}
\definecolor{color4}{RGB}{120,94,240}
\newcommand{\colorivname}{purple}
\definecolor{color5}{RGB}{220,38,127}
\newcommand{\colorvname}{red}

\newcommand{\colorsgx}{color3}
\newcommand{\colorsancus}{color2}
\newcommand{\colortz}{color5}
\newcommand{\colornative}{color4}

\usepackage{enumitem}
\newlist{paraenum}{enumerate*}{1}
\setlist[paraenum]{label=\emph{(\arabic*)}}

%% setting up side captions
\usepackage[rightcaption]{sidecap}
\makeatletter
\@ifdefinable\SC@listings@vpos{\def\SC@listings@vpos{b}}
\newenvironment{SClisting}{\SC@float[\SC@listings@vpos]{mylisting}}{\endSC@float}
\newenvironment{SClisting*}{\SC@dblfloat[\SC@listings@vpos]{mylisting}}{\endSC@dblfloat}
%% `sidecap` and `float` access the name of the new float differently.
%% You can't rely on `float` to pass `sidecap` the correct name.  So,
%% here we manually feed the correct name for the new float in a manner 
%% pleasing to `sidecap`.
\@namedef{mylistingname}{Listing}
\makeatother

%% creating figure command
\newcommand{\fig}[4]{%
  \begin{SCfigure}
    \centering
      \includegraphics[width=\textwidth]{#1}
    \caption[#2]{#3}
    \label{fig:#4}
  \end{SCfigure}%
}

\renewcommand\sidecaptionsep{5mm}
\sidecaptionvpos{figure}{t}

\newcommand{\sclist}[5]{%
  \begin{SCfigure}[.9]
    \centering
    {\begin{minipage}{#1\textwidth}
      {#2}
    \end{minipage}}
    \caption[#3]{%\protect\rule{0ex}{5ex}
     #4}
    \label{#5}
    \vspace{-3mm}
  \end{SCfigure}%
}

\newcounter{inlasscnt}
\renewcommand{\theinlasscnt}{{\bf A\arabic{inlasscnt}}}
\newcommand{\inlass}[2]{\refstepcounter{inlasscnt}\theinlasscnt\label{#1}:~{\it #2}}

\newcounter{ioasscnt}
\renewcommand{\theioasscnt}{{\bf IO\arabic{ioasscnt}}}
\newcommand{\ioass}[2]{\refstepcounter{ioasscnt}\theioasscnt\label{#1}:~{\it #2}}

\newcounter{dplasscnt}
\renewcommand{\thedplasscnt}{{\bf D\arabic{dplasscnt}}}
\newcommand{\dplass}[2]{\refstepcounter{dplasscnt}\thedplasscnt\label{#1}:~{\it #2}}
