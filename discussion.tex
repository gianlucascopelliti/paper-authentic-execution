\section{Discussion}
\label{sec:discussion}

\subsection{Integrity versus Confidentiality}
\label{sec:discussion:confidentiality}
%
We have focused our security objective on integrity and authenticity, and
an interesting question is to what extent we can also provide
confidentiality guarantees. It is clear that, thanks to the isolation
properties of protected modules and  to the confidentiality properties of
authenticated encryption, our prototype already provides substantial
protection of the confidentiality of both the state of the application as
well as the information contained in events. However, providing a formal
statement of the confidentiality guarantees offered by our approach is
non-trivial: some information leaks to the attacker, such as for instance
when (and how often) modules send events to each other.  This in turn can
leak information about the internal state of modules or about the content
of events. The ultimate goal would be to make compilation and deployment
fully abstract \cite{abadi} (indicating roughly that the compiled system
leaks no more information than can be understood from the source code), but
our current approach is clearly not fully abstract yet.
Hence, we decided
to focus on strong integrity first, and leave confidentiality guarantees
for future work.

There is also a wealth of orthogonal research aiming to protect network
information flows from being used by attackers to learn the system state by
means of, e.g, covert channels or routing protocols for mix networks
(cf.~\cite{shirazi2019anonymous} for an overview). The applicability of these
approaches depends on application configuration and available system resources
and may heavily impact availability guarantees. On light-weight \acp{TEE} such
as Sancus, and specifically in the presence of constraints on system resources
and power consumption, these additional protections are not applicable.

\subsection{Availability}
\label{sec:discussion:availability}

Our authentic execution framework focuses on the notion that if a valid input to
a module is received, we can deduce that it must have originated from an
authentic event at the source. By definition, this notion does \emph{not} give
any availability guarantees: only if we receive such a valid input can we make
claims about the authentic event, whereas nothing can be claimed otherwise.

However, there has been recent work on enforcing availability guarantees for
TEEs, specifically for Sancus~\cite{alder_2021_aion}. At its core, this related
work enables strict availability guarantees to enclaves on Sancus, such as that
an enclave will be executed periodically or be enabled to serve an interrupt
within a predictable time frame. This allows to make claims about how specific
enclaves will behave and \emph{when} these enclaves will react upon events they
receive. At the same time, it is still not possible to make any availability
claims about events that exceed the device boundary, i.e., events that cross
beyond the device itself and that are communicated over a shared untrusted
medium such as a network. Overall, it may be possible to make strong claims
about when and how fast a received event will be processed by an enclave, but it
is not possible to guarantee that an event will necessarily arrive over the
untrusted network.

Thus, we see availability as a complementary guarantee for authentic execution:
If an event arrives at the device boundary, earlier work such as
Aion~\cite{alder_2021_aion} can guarantee that this event can be handled within
a strict time frame, depending on the scenario's needs. Furthermore, the same
work can guarantee that if \emph{no} event is received within the expected time,
the enclave will also be able to react within a guaranteed time window. This may
be useful for a specific set of scenarios where disrupted availability could
lead to unwanted or dangerous outcomes. By employing availability as a
safeguard, the absence of events could, e.g., be used to trigger mode change and
activate a local control loop to emergency-shutoff equipment until an authentic
event is received.

While we do believe that this complementary notion of availability can have a
benefit to certain scenarios that utilize authentic execution, we leave the
combination of these two orthogonal research directions for future work.

\subsection{Hardware Attacks and Side-Channels}
%
Although hardware attacks and side-channels are explicitly ruled out by our
attacker model (\cref{concept:attacker}), it is necessary to discuss the impact
in case an attacker would have given access to such techniques. We leave an
analysis of our implementation for side-channels for future work.

An attacker that successfully circumvents the hardware protections on a node
would be able to manipulate and impersonate all modules running on \emph{that
node}.  That is, the attacker would be able to inject events into an application
but only for those connections that originate from the compromised node.  The
impact on the application obviously depends on the kind of modules that run on
the node.  If it is an output module, the application is completely compromised
since the attacker can now produce any output they want.  If, on the other hand,
it is one among many sensor nodes, that get aggregated on another node, the
impact may be minor.

Depending on the node type different attack vectors apply. A wealth of
literature targets Intel processors and SGX enclaves~\cite{brasser2017software,
gotzfried2017cache, weisse2018foreshadow, van2018foreshadow, van2020sgaxe} that
enable the extraction of cryptographic keys or other application secrets. On
embedded \ac{TEE} processors such as Sancus, many side-channels such as cache
timing attacks~\cite{cache-side-channels-practical,
practical-timing-side-channel-aslr} or page fault
channels~\cite{controlled-channel-attacks} are not applicable. Interrupt-based
side channels~\cite{van2018nemesis} do pose a challenge and have been mitigated
in orthogonal research in hardware~\cite{busi_2021securing} and in
software~\cite{winderix_2021_nemesis_def}.

