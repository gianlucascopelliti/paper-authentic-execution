\section{Introduction}
\label{sec:intro}

\acp{TEE} allow an application to execute in a hardware-protected environment,
often called \emph{enclave}. Enclaves are isolated and protected from the rest
of the system, ensuring strong confidentiality and integrity guarantees.
Cryptographic primitives and credential management, with keys that are unique
per enclave and that can only be used by that enclave, enable secure
communication and remote attestation; the latter is a mechanism to obtain
cryptographic proof that an application is running under enclave protection on a
specific processor.
%
Several \acp{TEE} are available in industry and research. Open-source \acp{TEE}
include Sancus and Keystone; proprietary options are, e.g., \ac{SGX} for Intel
processors, \ac{SEV} for AMD, TrustZone for ARM, and
others~\cite{maene:hardware}. Developing applications that execute on
heterogeneous \acp{TEE} is difficult, in particular for scenarios that combine
Internet-of-Things, Edge, and cloud hardware: each \ac{TEE} requires a
platform-specific software implementation, comes with different approaches to
key management and attestation, a different \ac{TCB} footprint, and provides
different hardware features and security guarantees.

The development of a distributed application composed of multiple modules
running on heterogeneous hardware is non-trivial by itself, but becomes an even
bigger challenge when the application has stringent security requirements that
demand the use of \ac{TEE} architectures. A developer needs to make choices as
to which security features are required for which components, adapt the code of
each component to multiple specific platforms, arrange for different deployment
and attestation strategies, and implement secure interaction between the
components. Open-source projects such as Open Enclave SDK and Google Asylo aim
to bridge the development gap between different \acp{TEE}.  However, software
engineers still need to account for the communication between different modules,
which has to be properly secured with cryptographic operations for data
encryption and authentication.  In particular, the responsibility for deploying
the distributed application, loading and attesting each enclave, establishing
session keys and secure connections between distributed components, is still
left to the application developer and operator. Overall, ensuring strong
security guarantees in distributed scenarios poses a challenge to the adoption
of \ac{TEE} technology.

This paper studies the problem of securely executing distributed applications on
such a shared, heterogeneous \ac{TEE}-infrastructure, while relying on  a small
run-time \ac{TCB}.  We want to provide the owner of such an application with
strong assurance that their application is executing securely.  We focus on
%
\begin{paraenum}
%
  \item \emph{authenticity} and \emph{integrity} properties of
%
  \item \emph{event-driven} distributed applications.
%
\end{paraenum}
%
For this selection of security property and class of applications, we specify
the exact security guarantees offered by our approach. We believe our approach
to be valuable for any kind of distributed application (event-driven or not). In
particular, our prototype supports arbitrary C and Rust code.

We distinguish physical events, such as sensor inputs or actuation, from logical
events that are generated and consumed by application components.
%
Roughly speaking, our notion of \emph{authentic execution} is the
following: if the application produces a physical output event (e.g., turns
on an LED), then there must have happened a sequence of physical input
events such that that sequence, when processed by the application (as
specified in the high-level source code), produces that output event.
%
This
notion of security is roughly equivalent to the concept of \emph{robust
safety} in later literature~\cite{abate2019journey}.
%
Our approach has been experimented with in previous work,
where we secure smart distributed applications in the context of smart
metering infrastructures~\cite{muehlber_smart_meter}, automotive
computing~\cite{vanbulck_2017vulcan}, and precision
agriculture~\cite{scopelliti2020thesis}, which we generalize and partly
formalize under the name \enquote{authentic execution}
in~\cite{noorman:authentic-execution}, and use as a training and tutorial
scenario to explain attestation and secure communication with heterogeneous
\acp{TEE}~\cite{muehlber_2018tutorial_enclaves}. 

The main contributions of this paper are:
%
\begin{itemize}
%
  \item We reflect on the design and implementation of an approach for authentic
execution of event-driven programs on heterogeneous distributed systems, under
the assumption that the execution infrastructure offers specific security
primitives -- standard enclaves~\cite{pma} plus support for secure
I/O~\cite{noorman_sancus2} (\cref{sec:design}); we comprehensively discuss
corner cases and hurdles regarding the use of secure I/O in distributed enclave
applications;
%
  \item We provide a revised open-source implementation of the approach for
Intel SGX, ARM TrustZone and Sancus, which supports software development in Rust
and C (\cref{sec:implementation}); we elaborate on implementation challenges
towards achieving comparable security guarantees in heterogeneous TEE
deployments;
%
  \item We conduct an end-to-end evaluation of a concrete application scenario,
considering the performance and security aspects of our framework
(\cref{sec:evaluation}), that considers dynamic system updates, and showing that
our approach allows for the deployment of complex distributed software systems
with a very small run-time application \ac{TCB};
%
  \item We design and implement a light-weight symmetric attestation scheme for
ARM TrustZone, inspired by and providing security guarantees comparable to
Sancus.
%
\end{itemize}

Our complete implementation of the authentic execution framework and the
evaluation use case are available at \implurl{}, a formalization and proof
sketch of our security guarantees is also available there.

