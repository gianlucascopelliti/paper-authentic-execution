\section{Authentic Execution of Distributed Applications}
\label{sec:design}

We outline our requirements for the infrastructure wrt. security features,
and show how these features are used effectively to accomplish our security
goals.

\subsection{Underlying Architecture: \acp{TEE}}
\label{design:tee}
%
Given the shared nature of the infrastructure assumed in our system model, we
require the ability to isolate \reactmods{} from other code running on a node.
Since an important non-functional goal is to minimize the \ac{TCB}, relying on a
classical omnipotent kernel to provide isolation is ruled out. Therefore, we
assume the underlying architecture is a \ac{TEE}~\cite{pma}.

\label{concept:isolation} While details vary between \acp{TEE}, isolation
of software modules is understood as follows: A module must be able to
specify memory locations containing data that are accessible by the
module's code only (\emph{data isolation}).  The code of a module must be
immutable and a module must specify a number of \emph{entry points} through
which its code can be executed (\emph{code isolation}).  For simplicity we
further assume that both a module's code and data are located in contiguous
memory areas called, respectively, its \emph{code section} and its
\emph{data section}. We expect \acp{TEE} to implement a $W\oplus X$ policy
so as to prevent code modification and code generation at runtime.

\label{concept:pma-compiler} We also expect the availability of a compiler
that correctly translates \acfp{SM} to the underlying
architecture. The input to this compiler is as follows:
%
\begin{paraenum}
%
  \item \label{compiler-input:entries} a list of entry point functions;
%
  \item \label{compiler-input:non-entries} a list of non-entry functions;
%
  \item \label{compiler-input:vars} a list of variables that should be
allocated in the isolated data section; and
%
  \item \label{compiler-input:consts} a list of constants that should be
allocated in the isolated code section.
%
\end{paraenum}
%
The output of the compiler should be an \protmod{} suitable for isolation on
the underlying architecture.

\label{concept:attestation}
%
Besides isolation, we expect the \ac{TEE} to provide a way to \emph{attest}
the correct isolation of an \ac{SM}.
%
\label{concept:pm-identity}
%
Attestation provides proof that an \ac{SM} with a certain identity has been
isolated on the node, where the \emph{identity} of an \ac{SM}, usually based on a measurement of the deployed binary code, should give
the deployer assurance that this \ac{SM} will behave as the corresponding
source code module.

\label{concept:secure-communication}
%
After enabling isolation, the \ac{TEE} should be capable of establishing a
confidential, integrity protected and authenticated communication channel
between an \ac{SM} and its deployer.  Although the details of how this works may
differ from one \ac{TEE} to another, for simplicity we assume the \ac{TEE}
establishes a shared secret between an \ac{SM} and its deployer and provides an
authenticated encryption primitive.  We refer to this shared secret as the
\emph{module key}.  The authentication property of the communication channel
refers to an \ac{SM}'s identity and hence to attestation.  Thus, the \ac{TEE}
ensures that if a deployer receives a message created with a module key, it can
only have been created by the corresponding, correctly isolated, \ac{SM}.

\subsection{Mapping \reactmods{} to \protmods{}}
\label{concept:compilation}
%
To map a \reactmod{} to an \protmod, we use the following procedure.  First,
each of the \reactmod's inputs and outputs is assigned a unique \emph{I/O
identifier}, and each of the connections between \reactmods{} is assigned a
unique \emph{connection identifier}. The format of these identifiers is
unimportant, provided that they uniquely specify a particular input/output or a
particular connection, respectively. By having a clear distinction between I/O
identifiers and connection identifiers, many-to-many connections are supported,
which means that:
%
\begin{paraenum}
  %
    \item a \reactmod's output can be connected to multiple inputs of
    other \reactmods{} (e.g., a keyboard that sends key press events to both
    a key-logger and an LCD screen), and
  %
    \item a \reactmod's input can be reached by
    multiple outputs of other \reactmods{} (e.g., a key-logger that records
    key presses of two different keyboards).
  %
  \end{paraenum}

Second, a table (\conntab) is added to the \protmod's variables that maps
connection identifiers to a \conn{} data structure, such that every connection
has one entry associated with it. These entries will be initialized to all zeros
by the underlying architecture, which is interpreted as an unestablished
connection. For establishing a connection, an entry point is generated
(\setkey).  This entry point takes a connection identifier, an I/O identifier
and a key -- encrypted using the module key -- as input and updates the
corresponding mapping in \conntab{} (\cref{code:setkey}). Since every connection
needs to be protected from reordering and replay attacks, the \conn{} structure
also includes a nonce.

\sclist{0.5}{%
  \lstinputlisting[language=Python]{setkey.py}}{}%
  {Pseudocode of the \setkey{} entry point. Note that \unwrap{} uses the module
  key to decrypt the payload and throws an exception if the operation failed
  (i.e., the payload's \acs{MAC} is incorrect).\label{code:setkey}}%
  {code:setkey}
%
%
\sclist{0.5}{%
  \lstinputlisting[language=Python]{handleinput.py}}{}%
  {Pseudocode of the \handleinput{} entry point. Erroneous accesses to the
  tables as well as errors during \unwrap{} cause exceptions. Thus, these
  events, as well as those for which no input key has been set, are ignored.
  \unwrap{} takes a key and the expected associated data as arguments.}%
  {code:handleinput}
%
%
\sclist{0.5}{%
  \lstinputlisting[language=Python]{handleoutput.py}}{}%
  {Pseudocode of the generated output wrapper. Since the compiler generates
  calls to this function that cannot be called from outside the module, the
  connection identifier is always valid and no error checking is
  necessary.}%
  {code:output} 
%
%
\sclist{0.5}{%
  \lstinputlisting[language=Python]{attest.py}}{}%
  {Pseudocode of the \attest{} entry point. Since the attestation of an
  \protmod{} is \ac{TEE}-specific, we use a high-level function \genattevid{} to
  retrieve the attestation evidence. The actual implementation of this function
  depends on the \ac{TEE} used.}%
  {code:attest} 

Third, all the module's event handlers are marked as non-entry functions.  A
callback table (\cbtab) is added to the \protmod's constants, mapping input
identifiers to the corresponding event handlers.  This table is used by the
entry point \handleinput, which is called when an event is delivered to the
\protmod. \handleinput{} takes two arguments: a plain-text connection identifier
and an encrypted payload.  If \conntab{} has a connection for the given
identifier, its key is used to decrypt the payload (using the \emph{expected}
nonce as associated data), which is then passed to the callback function stored
in \cbtab, retrieved using the I/O identifier stored in the Connection
structure.  If any of these operations fails, the event is ignored
(\cref{code:handleinput}).  From a programmer's perspective, an input callback
will only be called for events generated by entities with access to valid
connection keys.

\label{concept:output-wrapper}
%
Fourth, each call to an output is replaced by a call to a non-entry wrapper
function \handleoutput{}.  This function takes a connection identifier and a
payload, encrypts the payload together with the current nonce using the
corresponding connection key, then publishes the encrypted event (via
\handlelocalevent, cf.~\ref{concept:event-manager}). If the output is
unconnected, the output event is dropped (\cref{code:output}).

Fifth, an entry point \attest{} is generated, which will be called by the
deployer to attest the \protmod{}. This function takes a challenge as input and
returns an \emph{attestation evidence} as response. The attestation procedure
and the content of the attestation evidence may vary from one \ac{TEE}
technology to the other; nonetheless, the input challenge is typically used as a
freshness parameter to prevent replay attacks. On some \acp{TEE}, the
attestation process also includes the generation of a shared secret between the
\protmod{} and the deployer, which will be used as the module key
(\cref{code:attest}).

To conclude, the following \protmod{} definition is given as input to the
\ac{TEE} compiler:
%
\begin{paraenum}
%
  \item \setkey, \attest{} and \handleinput{} as entry points;
%
  \item input event handlers and \handleoutput{} as non-entry functions;
%
  \item \conntab{} as a global variable; and
%
  \item \cbtab{} as a constant.
%
\end{paraenum}
%
\Cref{flt:compiled-pm} shows the compiled memory layout of one of the
example modules.

\sclist{0.5}{\maxsizebox{.75\linewidth}{!}{\begin{tikzpicture}
  \colorlet{cdata}{color3!80}
  \colorlet{ccode}{color2}

  \tikzstyle{memloc}=[draw, text width=3cm, align=center]
  \tikzstyle{freemem}=[memloc]
  \tikzstyle{code}=[memloc, fill=ccode]
  \tikzstyle{data}=[memloc, fill=cdata]

  \tikzstyle{brace}=[decorate, decoration={brace, amplitude=4pt}]
  \tikzstyle{lbrace}=[brace, decoration={mirror}]
  \tikzstyle{rbrace}=[brace]
  \tikzstyle{bracelbl}=[midway, text width=2cm, font=\small, inner sep=5pt]
  \tikzstyle{lbracelbl}=[bracelbl, left, align=right]
  \tikzstyle{rbracelbl}=[bracelbl, right]

  \matrix[matrix of nodes] {
    |[freemem]             | $\cdots$                             \\
    |[code] (Start code)   | Standard entry stub                  \\
    |[code] (Start entries)| \setkey								       			 	\\
    |[code] (Attest)       | \attest								       			 	\\
    |[code] (End entries)  | \handleinput   					       		 	\\
    |[code] (Start funcs)  | \handleoutput  				       		   	\\
    |[code] (Start eh)     | \code{Sensor}					       		 	  \\
    |[code] (End eh)       | \code{Tick}					       		 	    \\
    |[code] (CbTable)      | \cbtab         				       	     	\\
    |[code] (Start const)  | \code{SATURATED}                     \\
    |[code] (End const)    | \code{MAX}                           \\
    |[freemem]             | $\cdots$                             \\
    |[data] (CompData)     | Standard runtime data (stack,\ldots)	\\
    |[data] (ConnTable)    | \conntab                             \\
    |[data] (P1)           | \code{flooded}                       \\
    |[data] (P2)           | \code{count}                         \\
    |[freemem]             | $\cdots$                             \\
  };
  
  \draw[lbrace] (Start entries.north west) -- (End entries.south west)
                node[lbracelbl] {\ref{compiler-input:entries} Entry points};
  \draw[lbrace] (Start funcs.north west) -- (End eh.south west)
                node[lbracelbl] {\ref{compiler-input:non-entries} Non-entry functions};
  \draw[lbrace] (CbTable.north west) -- (End const.south west)
                node[lbracelbl] {\ref{compiler-input:consts} Constants};
  \draw[lbrace] (ConnTable.north west) -- (P2.south west)
                node[lbracelbl] {\ref{compiler-input:vars} Variables};
  \draw[rbrace] (Start code.north east) -- (Start funcs.south east)
                node[rbracelbl] {Generated};
  \draw[rbrace] (Start eh.north east) -- (End eh.south east)
                node[rbracelbl] {Event handlers};
  \draw[rbrace] (CbTable.north east) -- (CbTable.south east)
                node[rbracelbl] {Generated};
  \draw[rbrace] (Start const.north east) -- (End const.south east)
                node[rbracelbl] {Constants};
  \draw[rbrace] (CompData.north east) -- (ConnTable.south east)
                node[rbracelbl] {Generated};
  \draw[rbrace] (P1.north east) -- (P2.south east)
                node[rbracelbl] {Globals};
\end{tikzpicture}
}}{}{%
  Memory layout of the compiled version of module \module{FloS1} in application
  \appvio{} (\cref{code:appvio}). The code- and data sections are shaded in
  \coloriiname{} and \coloriiiname{}, respectively. The numbers on the left
  labels correspond to the compiler inputs, while the right labels indicate
  whether parts are implicitly generated by the compiler or correspond to source
  code.}{flt:compiled-pm}

\subsection{Untrusted Software on the Nodes}
%
\label{concept:untrusted-sw}
%
To support the deployment of modules and the exchange of events between modules,
untrusted (and unprotected) software components need to be installed on each
node, as outlined below.

\subsubsection{Module Loader.} \label{concept:module-loader}
%
The module loader is an untrusted software component running on every node.
The module loader provides services for external entities to interact with
modules on a node.  To this end, the module loader listens for two types of
remote requests: \loadmodule{} and \callentry.  \loadmodule{} takes a
compiled \protmod{} as input, loads it into the \ac{TEE} and returns the
module's unique identifier together with all information necessary for
attestation and module key establishment.  What exactly this information is
and how the attestation and key establishment is performed is specific to
the used \ac{TEE}.  \callentry{} takes an \protmod's identifier, the
identifier of an entry point and potentially some arguments and calls the
entry point with the given arguments.

\subsubsection{Event Manager.} \label{concept:event-manager}
%
The event manager is another untrusted software component running on every
node that is used to route events from outputs to inputs.  It recognizes
three types of requests: \addconnection, \handlelocalevent{} and
\handleremoteevent.  A deployer can invoke \addconnection{} remotely to
connect the output of a module to the input of another.  How exactly inputs
and outputs are identified is implementation specific but it will in some
form involve specifying
%
\begin{paraenum}
%
  \item a node address (e.g., an IP address);
%
  \item an \protmod{} identifier; and
%
  \item a connection identifier.
%
\end{paraenum}
%
As will become clear later, \addconnection{} only needs to be called on the
event manager of the node where the output \reactmod{} is deployed.

\handlelocalevent{} is used by modules to publish an event; i.e., inside the
output wrappers. The arguments are the module and connection identifiers and the
event payload. Based on the identifiers, the event manager looks up the
destination event manager and invokes its \handleremoteevent{} API, providing
the identifiers of the input to which the request should be routed.  For a
\handleremoteevent{} request, the event manager will check if the destination
module exists and, if so, invoke its \handleinput{} entry point, passing the
connection identifier and payload as arguments.

\subsubsection{Implementation.}
%
Although we introduced the module loader and the event manager as two separate
software components, in our prototype we merged the functionalities of both in
the same application, which we simply call the \emph{Event Manager}. This design
choice was made for the sake of simplicity: the actual implementation of such
logical components is unimportant as long as each node provides the
functionalities described above.

\subsection{Physical Input and Output Channels}
%
\label{concept:protected-drivers}
%
We assume that the infrastructure offers physical input and output channels
using \emph{protected driver modules} that translate application events into
physical events and vice versa. Driver modules control these physical events --
sensing and actuation -- within the boundaries of the processor executing the
driver. For input channels, these modules generate events that correspond to
physical events and provide a way for application modules to authenticate the
generated events. 
%
\label{concept:output-attestation}
%
For output channels, a driver module (\module{D}) must have exclusive access to
its I/O device ($D$) and allow an application module (\module{A}) to take
exclusive access over the driver.  That is, the driver will only accept events
-- and hence translate them to physical events -- from the application module
currently connected to it. Below, we put explicit requirements on the implementation of secure I/O and mark them for later reference.

First, we demand that \ioass{ass:phyconfig}{the infrastructure
  provider configures the physical I/O devices as expected}, i.e., the
desired peripherals are connected to the right pins and thus mapped to the
correct \ac{MMIO} addresses in the node. Since misconfiguration in the physical
world cannot be detected by remote attestation, we need to require that I/O
devices are set up correctly.

However, \ioass{ass:exclusive}{the infrastructure must provide a
  way for the deployer of \module{A} to attest that it has exclusive access to
  the driver module \module{D} and that \module{D} also has exclusive access to
  its I/O device $D$}.~ The deployer must be able to attest \module{D}
to ensure that it indeed only accepts events from the module currently having
exclusive access and that it does not release this exclusive access without
being asked to do so by the module itself.

We also need to require that, \ioass{ass:startup}{after a
  microcontroller is turned on, only authenticated driver modules may take
  control of I/O devices. Once control is taken it is exclusive and never
  released}. This prevents attackers from taking direct control of
output devices after a node resets. Driver modules are thus part of the trusted
computing base and their module keys are only known to the infrastructure
provider who authorizes exclusive connections to driver modules upon request
from the deployer and keeps track of the ``ownership'' of the driver modules.

Finally, we demand that \ioass{ass:io}{output driver modules do not produce
outputs unless requested by the application module, while input driver modules
only generate outputs to application modules that correspond to physical inputs.}

To summarize, we have modeled physical \emph{output} channels in such a way that:
%
\begin{enumerate}
  %
    \item  At startup, authenticated output driver modules take
      ownership over the (correct) output devices. Unauthorized access and
      connection attempts from other software are forbidden;
  %
    \item Application modules can take ownership of an output driver module only
      if the latter is not already connected to other modules, and only if
      authorized by the infrastructure provider;
  %
    \item After connecting to an output driver module, an application module
      retains exclusive access to the output device until explicitly released.
       Thanks to point (1), exclusive access is retained even across
      resets, although all connections need to be re-established.
  %
  \end{enumerate}

 For input modules, exclusive access is not strictly
required.~ However, exclusive access to input devices can be
configured in the same way as for outputs, if desired. These conditions lead to
the following security properties:
%
\begin{itemize}
\item \emph{Assume that for a time interval $T$ the infrastructure provider only
  ever authorizes application module \module{A} to take control of
  device $D$ via driver module \module{D}. If the deployer attests within $T$
  that \module{A} has taken exclusive access over output driver module
  \module{D}, then from that moment until the end of $T$ every output from $D$
  can be explained alone by the code semantics of \module{A} and \module{D} in
  relation to the inputs received by \module{A}, i.e., no other module may cause
  outputs of $D$.}
\item \emph{If the deployer attests that \module{A} has taken access over an
  input driver module \module{D}, then from that moment on until the access is
  released any output from \module{A} can be explained alone by the code
  semantics of \module{A} and \module{D} in relation to the physical inputs to
  $D$ and any other inputs received by \module{A}, i.e., an attacker cannot
  spoof inputs to $D$ for \module{A}.}
\end{itemize}
%
In practice, I/O devices may produce initialization outputs when a
driver module takes control of them after a reset. Handling these outputs
is parts of the specificities of a driver implementation, which we
ignore where for simplicity.

\subsection{Deployment} \label{concept:deployment}
%
Deployment is the act of installing application modules on their nodes and
setting up connections between outputs and inputs. As a requirement for
deployment, \dplass{ass:deploychan}{the channel between deployer and
infrastructure provider is assumed to be secure}, whereas communication with
modules is performed over an untrusted network.

In phase 1, the deployer requests access to the driver modules connected to
physical I/O devices from the infrastructure provider who provides these
modules. For output devices, the deployer requests \emph{exclusive} access,
while for input devices the deployer can choose between exclusive or shared
access.  Before granting such access, \dplass{ass:attest}{the infrastructure
provider needs to ensure the authenticity of the driver module controlling the
I/O device}, e.g., via attestation.  If, for any reason, the infrastructure
provider cannot give access to some I/O devices (e.g., because an output device
is already taken by someone else), the deployment is aborted.  Otherwise, the
deployer receives connection keys from the infrastructure provider (one for each
I/O device), along with the guarantee that driver modules have correctly taken
control of the physical I/O devices (\cref{concept:protected-drivers}).

Phase 2 consists of deploying the application on the infrastructure. First, the
deployer starts by compiling each source module into a loadable image.  Second,
the deployment descriptor is used to find the node on which the module should be
deployed and sends its module loader a \loadmodule{} request. The deployer then
performs the \ac{TEE}-specific method of attestation calling the \attest{} entry
point of each \protmod, eventually setting up the module key. At the end of this
step, the deployer has a secure communication channel with each of its deployed
\reactmods. Finally, the deployer sets up the connections between modules, as
well as the connections to the driver modules. Regarding the former, the
deployer generates a unique connection key and sends it to both endpoints of the
connection; for each endpoint, such key is encrypted using the module key and
passed to the \setkey{} entry point using the \callentry{} API. Concerning the
latter, instead, connections are established via a similar interface, using the
keys obtained in phase 1.

\paragraph{Security Discussion}
\label{concept:security-discussion}
%
During phase 1, the infrastructure provider plays an important role as a trusted
party, providing exclusive access to driver
modules. \dplass{ass:driverkeys}{It is required that the provider
  does not leak the driver module keys and that exclusive access to a device is
  reserved for the deployer until they request to release it or a time T has
  passed (to which both the deployer and the provider have agreed
  upon)}.~ In other words, no one else can acquire credentials to
connect to the driver module during that time.

If a reset on a node N occurs, then all driver modules belonging to that node
need to be re-initialized, and exclusive access re-configured (if previously
set). Concerning the latter, \dplass{ass:driverreplay}{the infrastructure must
provide replay protection for messages exchanged in the protocol to establish
exclusive driver access}, otherwise replay attacks might cause exclusive access
to be unintentionally set up for modules that formerly had access, but should
not have it anymore.

Concerning application modules instead, a reset on node N would cause all
modules of that node to be re-deployed, re-attested, and connections for them to
be re-established. Here, an attacker could technically mount a sophisticated
replay attack to restore a previous configuration of an application; however, if
replay protections for driver modules are correctly in place (as discussed
above), the attacker would not be able to trigger any illicit or stale outputs. 

Note that the strong requirements on physical output drivers do not apply for
virtual outputs to software endpoints, such as secure databases or nodes
belonging to a trusted party like the deployer. Such receivers can verify the
authenticity and freshness of outputs via cryptographic means and reject
replayed messages automatically, which is not possible for physical outputs to
the real world.

\subsection{Security Argument}
%
Our goal is to ensure that all physical output events can be explained by the
application's source code and the observed physical input events. More
precisely:
%
\label{concept:security-property} %
%
\emph{ Consider a time frame starting at the end of phase 1 of deployment
  (\cref{concept:deployment}), and ending at a point where the deployer releases
  exclusive access to any of the output devices they control. Assume the
  deployer has observed a specific sequence of physical output events on one of
  these devices \dev{O} in the considered time frame, then there have been
  contiguous sequences of physical input events on the input devices connected
  to the application such that the observed outputs follow from these inputs
  according to the application source code semantics.}

As an example, consider \appvio{} (\cref{code:appvio}). If, after the
application has been deployed, the water supply is turned off, then there must
have been physical input events that caused the field to flood.

Output events can only be produced by the application's \acp{SM}; the assumption
of a correct compiler and the successful attestation then lead to the desired
property.
%
 Due to requirements \ref{ass:phyconfig}-\ref{ass:io},
%
\begin{paraenum}
%
  \item a physical output event can only be produced by the corresponding
device (\dev{O});
%
  \item output drivers have exclusive access to their device; and
%
  \item an \ac{SM} (\module{O}) has exclusive access to the driver;
%
\end{paraenum}
%
thus only \module{O} can initiate physical outputs on \dev{O}.  Even if
exclusive access is lost, e.g., due to a reset of the corresponding node
triggered by an attacker, the infrastructure prevents that any other module can
take control of \dev{O} and produce rogue outputs, thanks to requirements
\ref{ass:deploychan}-\ref{ass:driverreplay}.
%
The construction of \acp{SM} ensures that a module can only be invoked through
its two entry points.  Of these, only \handleinput{} can result in output events
(\cref{code:setkey,code:handleinput}).  Since \handleinput{} authenticates its
input, output events are always the result of correct input events. Finally, as
our deployment scheme only allows for two types of correct input events,
physical input events and outputs from other modules, our security property
follows.

\subsection{Software Updates and Re-Deployment}
\label{concept:updates}

In \cref{concept:deployment}, we described a \emph{static} deployment where the
whole application is deployed at once; however, it is also possible to deploy
application modules dynamically. Specifically in scenarios where the the
configuration of an application may change during its lifetime due to the
reconfiguration, addition, or removal of sensors or processing nodes, or the
deployment of additional applications on existing infrastructure, software
updates become essential. In our smart irrigation system from
Figure~\ref{fig:scenario}, this may, e.g., happen with changing environmental
conditions or when plots are re-allocated for different crops.

Two common scenarios that may occur at runtime are software updates and
re-deployments. The former consists of deploying an updated version of an
\acp{SM} that, e.g., provides new functionality or fixes some bugs; the latter
does not necessarily require code changes but may be needed when an \ac{SM}
crashes or needs to be migrated to another node. 
%
Our design for software updates and re-deployments consists of three steps:

\begin{enumerate}
  \item \emph{Deployment of the new \ac{SM}.} In the first step, the new \ac{SM}
  is compiled, deployed on the infrastructure alongside the already running
  \acp{SM}, and finally attested. If any of these operations fails, the update
  is aborted. Note that the new \ac{SM} will not process any events until
  connections are configured through its \setkey{} entry point (step 3).
  \item \emph{Deactivation of the old \ac{SM}.} In this step, the old \ac{SM} is
  removed from the infrastructure. This is the point in time from which the
  application will start suffering from connectivity loss, until the new \ac{SM}
  will be completely configured. Optionally, state may be transferred to the new
  \ac{SM} before taking down the old one.
  \item \emph{Re-establishment of connections.} Here, all the connections
  associated to the old \ac{SM} must be re-configured in order to point to the
  new \ac{SM}. This process involves calling the \setkey{} and \addconnection{}
  entry points on the involved \acp{SM} and event managers, as for a static
  deployment (\cref{concept:deployment}); each connection retains its connection
  identifier, but keys are rotated for security reasons. Similarly, connections
  between the \ac{SM} and I/O devices must be re-configured as well, if any
  (\cref{concept:protected-drivers}). At the end of this step, the update will
  be completed.
\end{enumerate}

We evaluate the impact of software updates in Section~\ref{eval:microb-update}.
Note that actions 2 and 3 can also be executed in reverse order, if desired.
Re-establishing the connections before disabling the old \ac{SM} would result in
a smaller availability disruption, but it may also cause inconsistent state if
events are generated during the update. For example, some events might be
processed by the old \ac{SM} and some other events by the new \ac{SM}, causing
undefined behavior. As our framework mainly focuses on integrity rather than
availability, we should prevent such inconsistencies from occurring, for example
by implementing a sort of synchronization mechanism between old and new \acp{SM}
during the update. However, this would increase the complexity of our framework
without providing substantial benefits compared to our original update strategy.
Hence, we decided not to investigate this option further.

Transferring the state from the old to the new \ac{SM} may be done in several
ways. For instance, the \emph{sealing} feature of certain \acp{TEE} (such as
Intel \ac{SGX}) allows an \ac{SM} to securely store data persistently on disk,
although it requires the new \ac{SM} to be deployed on the same node as the old
one. Our design already provides some basic support for state transfer by
creating a \emph{temporary connection} between the two \acp{SM}. In particular,
a \emph{transfer} output in the old \ac{SM} may be connected to a \emph{restore}
input in the new \ac{SM}, then the state migration can be triggered by calling a
\emph{save} entry point in the old \ac{SM} that generates the \emph{transfer}
event. This would allow state to be securely transferred between the two
modules, even in different nodes. Of course, the deployer still needs to
manually implement the \emph{save} and \emph{restore} functions in order to
specify what data needs to be migrated. This protocol, however, requires the old
\ac{SM} to be up and running at the time of the update, and therefore cannot be
applied if the module was disabled before, e.g., due to a crash or a reset on
the node.


